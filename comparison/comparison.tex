\documentclass[11pt,a4paper]{article}
\usepackage{enumitem}
\usepackage[utf8]{inputenc}
\usepackage{nameref}
\usepackage[margin=3cm]{geometry}
\usepackage[english]{babel}
\usepackage{csquotes}
\usepackage[style=numeric, sorting=none]{biblatex}
\usepackage{amsmath}
\usepackage{graphicx}
\usepackage[colorlinks=true, allcolors=blue]{hyperref}
\hypersetup{
    colorlinks,
    linkcolor=black,  
    urlcolor=blue,
    citecolor=blue
}
\setlength{\marginparwidth}{2cm}
\usepackage{comment}
\usepackage{float}
\usepackage{booktabs}

\bibliography{references}
\author{Bruno Sánchez Gómez}
\date{\today}


\begin{document}

\begin{titlepage}
    \centering
    \vspace*{2cm}
    {\Huge \bfseries MBM Essay 3: Comparison \par}
    \vspace{2cm}
    {\huge From Neural Inspiration to Neural Explanation: The Symbiotic Journey of AI and Neuroscience \par}
    \vspace{10cm}
    {\large \textbf{Bruno Sánchez Gómez} \par}
    \vfill
    {\large \today \par}
\end{titlepage}

\section{Introduction}

In this essay, I offer a comparative look at two preceding works: Essay 1~\cite{essay1}, which I wrote through traditional research and writing, and Essay 2~\cite{essay2}, an essay on the same topic generated by a Large Language Model (LLM). My main goal here is to explore the nuances of these distinct approaches to creating content. I will look closely at how they differ in method, discuss the strengths and weaknesses of both human and AI writing, and evaluate their overall quality.

This essay is structured to give a clear overview of the whole experiment. I will start by describing how I wrote Essay 1, covering the research, planning, and drafting. Next, I will explain the prompting strategy I used for the LLM to generate Essay 2, including the adjustments I made along the way. The heart of the essay will be a direct comparison, looking at both the writing process and the final quality of each. This will lead to a discussion of the advantages and disadvantages I observed. Finally, I will bring these observations together and reflect on how academic writing is changing in an age of AI.

\section{Essay 1: Writing Process}

Writing Essay 1~\cite{essay1}, titled ``From Neural Inspiration to Neural Explanation: The Symbiotic Journey of AI and Neuroscience'', was a multi-stage process that took several days of focused work. Here is a summary of my process:

\begin{itemize}
    \item \textbf{Initial Topic Exploration and Refinement:} I was initially interested in exploring the intersection of AI and neuroscience. The first working title I had was ``Reverse-Engineering the Brain through Explainable AI''. However, my initial research showed there were not many papers explicitly detailing the use of Explainable AI (XAI) for direct neuroscientific discovery. This made me broaden the scope to a more general but still connected theme, focusing on the symbiotic relationship between AI and neuroscience. Still, I made a conscious effort to keep a strong focus throughout the essay on the potential of AI, and specifically XAI, as a powerful tool for understanding the brain.

    \item \textbf{Structuring the Historical Narrative:} For the first part of the essay, the historical review, I drafted a timeline to chart the co-evolution of AI and neuroscience. This meant identifying key milestones, influential figures, and important publications that shaped both fields and their mutual influences.

    \item \textbf{Researching Core Concepts:} I spent a good deal of effort understanding how XAI addresses the ``black box'' problem in many complex AI models, and more critically, how these XAI methodologies are beginning to be, or could be, applied to derive neuroscientific insights. This meant looking into various XAI techniques and how they relate conceptually to brain research.

    \item \textbf{Investigating the State of the Art:} I looked into current frontiers in AI, especially those with strong parallels or direct uses in neuroscience. This included exploring generative models, advancements beyond supervised learning, the emerging field of ``Neuro-AI'', and the use of LLMs in simulating cognitive processes.

    \item \textbf{Considering Broader Implications:} It was also important to address the challenges, limitations, and ethical issues associated with employing AI in neuroscience, recognizing the profound societal and philosophical questions involved.

    \item \textbf{Synthesis and Composition:} The final stage was to bring all this information together, supported by figures I found online, into a coherent narrative. This was an iterative process of drafting, refining, and making sure the arguments flowed logically, which took up the most time. This careful process, though demanding, allowed me to dive deep into the subject.
\end{itemize}

\section{Essay 2: Prompting Strategy}

When generating Essay 2~\cite{essay2}, my main goal was to get an LLM to produce an essay as similar as possible to my Essay 1 within \textit{one single, comprehensive prompt}. This required some experimentation with different LLMs and how I prompted them.

\begin{itemize}
    \item First, I gave Grok (xAI) the PDF of Essay 1 and asked it to write a similar essay. The result was acceptable but underwhelming: Grok copied the general structure, but each section was just a brief summary of my original, lacking depth.

    \item Then, I tried Gemini 2.5 Pro (Google), giving it my PDF in the same way. This time, the result was surprisingly good. Gemini produced an essay that was very close to mine in structure, content, style, and even tried to copy some of the formatting. This really showed what Gemini 2.5 Pro could do. However, it also made me rethink my objective. I realized I was not just interested in whether an LLM could copy an existing work if given the original. Instead, I wanted to see if an LLM could generate a complex, research-based essay from a fairly general prompt, simulating a situation where the user had not done all the prior research and writing.

    \item So, the final prompt I used for Essay 2 was designed to give a thematic and structural outline without providing specific content. The prompt was as follows:
    \vspace{-0.5em}
    \begin{center}
    \setlength{\fboxsep}{5pt}
    \setlength{\fboxrule}{1pt}
    \fbox{%
        \parbox{0.95\linewidth}{%
                Write a very detailed essay (10-15 pages) with the title ``From Neural Inspiration to Neural Explanation: The Symbiotic Journey of AI and Neuroscience''. You should cover:
                \begin{itemize}
                    \item a historical review of the joint evolution of AI and Neuroscience
                    \item an analysis on Explainable AI and how it connects to Neuroscience
                    \item the current state of the art in Explainable AI for Neuroscience and its challenges, limitations and ethical considerations
                \end{itemize}
                Write it in LaTeX.
        }
    }
    \end{center}
    This prompt asked the LLM to generate the essay from its own knowledge. The only changes I made to the raw LaTeX output were adding the standard title page and adjusting the font and page formatting for consistency with the other essays. This approach resulted in Essay 2, which, while similar in its main theme and structure to mine, naturally had differences in content, specific examples, and style.
\end{itemize}

\section{Comparative Analysis}

Here, I will directly compare Essay 1~\cite{essay1} and Essay 2~\cite{essay2}, focusing on two main things: how they were written and the quality of the final essays.

\subsection{Comparison of Writing Processes}

The way Essay 1 and Essay 2 were created was vastly different, highlighting the fundamental distinctions between human and LLM-driven writing.

\begin{itemize}
    \item \textbf{Essay 1 (Human):} Completing Essay 1 was an intellectually demanding and time-consuming task that took several days. It required a lot of research, going through academic papers, articles, and books to find relevant information. After that came a phase of critical analysis and synthesis, where I had to connect different pieces of information, form arguments, and structure them into a coherent story. The writing itself was an iterative process of drafting, redrafting, and refining, which took considerable mental effort. However, this challenging process was also very rewarding because it helped me understand the subject thoroughly and develop my own perspectives and arguments.

    \item \textbf{Essay 2 (LLM):} In contrast, the ``writing'' process for Essay 2 was much faster and required far less direct intellectual work from me. The main effort involved less than two hours of experimenting with different LLMs and prompts, leading to the single prompt I described earlier. Once the prompt was submitted, the LLM generated the entire essay draft in a matter of minutes. In this case, I did not do any primary research or content generation for this particular essay. While the LLM used its vast training data, my role was just to provide the prompt and then lightly edit the formatting. It is clear that if I had not already done the research for Essay 1, what I learned from just generating Essay 2 would have been much more superficial.
\end{itemize}

\subsection{Comparison of Resulting Essay Quality}

Both essays, despite being created differently, reached a good level of quality, but they have distinct features and some limitations.

\begin{itemize}
    \item \textbf{Essay 1 (Human):} I believe Essay 1 is a well-structured and coherent piece that goes into the topic in considerable depth. It systematically covers the history of AI and neuroscience, explores how XAI can bridge understanding, and discusses the current state of the art, including challenges and ethical issues. The arguments, I hope, are clearly expressed, and the use of external sources and figures (found online) is intended to support the narrative and help the reader understand. The style reflects my personal voice and approach to academic writing.

    \item \textbf{Essay 2 (LLM):} The LLM-generated Essay 2 is also remarkably well-structured and coherent, covering the same main topics from the prompt. It features a particularly thorough and detailed historical review. A key feature is how consistently and explicitly it sticks to the central theme; for almost every point, the essay carefully explains its relevance to both AI and neuroscience. While this keeps it focused, it also subtly shows how the LLM sticks to the prompt's limits, perhaps making it feel a bit more constrained than a human writer, who might weave in related thoughts or personal emphasis. The writing is high quality, formal, and very well done.
    However, two critical problems came up in Essay 2 that, I think, highlight why human oversight is still necessary:
    \begin{enumerate}
        \item \textbf{Reference Hallucinations:} Many of the bibliographic references in Essay 2 did not actually exist. The LLM seems to have hallucinated plausible-sounding academic papers, complete with authors, titles, and publication details. This is a serious flaw that undermines the academic integrity of the work if left uncorrected.
        \item \textbf{Absence of Illustrative Figures:} Essay 2 did not include any figures, diagrams, or visual aids. This is a basic limitation of current text-based LLMs: they cannot create or find relevant images on their own to go with the text. Figures, as used in Essay 1, can be crucial for clarifying complex concepts and making an essay more engaging.
    \end{enumerate}
\end{itemize}

Overall, it is hard to deny that when you look at the quality-to-effort ratio, the LLM makes a strong case. It can produce a high-quality, detailed, and well-written essay very quickly with minimal human input beyond the initial prompt. Still, problems like fake references and the lack of visuals show that, for now, human supervision, editing, and fact-checking are essential to ensure accuracy, completeness, and scholarly value of the produced content. LLMs are undoubtedly powerful tools for assistance, and ignoring their potential would be as outdated as avoiding search engines like Google for research. However, relying on them for entirely independent, unsupervised work, especially in academic contexts, is still some years away.

\section{Conclusion}

This comparison has highlighted the different paths and results of human versus LLM-driven essay writing. We have seen that my process for writing Essay 1~\cite{essay1} led to significant learning but involved deep research, critical thinking, and iterative writing, which took considerable time and mental effort. In sharp contrast, generating Essay 2~\cite{essay2} with an LLM was extremely fast and required minimal direct effort from me beyond crafting the prompt and it produced a remarkably articulate and well-structured piece, yet would have resulted in minimal learning, had it been the only approach.

The final essays, while both about the symbiotic journey of AI and neuroscience, I believe showed a reflection of how they were made: Essay 1 reflected my personal research journey and arguments, supported by visual aids; meanwhile, Essay 2 demonstrated the LLM's ability to recall comprehensive information and structure it coherently, but it also revealed critical flaws like fake references and no visual content.

The undeniable efficiency and high baseline quality of LLM-generated text suggest that if the only goal is to quickly produce a well-written document on a topic, then an LLM combined with some careful human supervision for fact-checking and visual refinement, is a powerful approach. Therefore, in an age where LLMs can so skillfully assemble information, the traditional, ``manual'' way of crafting a research essay or similar text gains a new, perhaps deeper, significance. It moves from being seen as just a means to an end (the finished document) to a more conscious, voluntary process. This human-led effort becomes valued not just for its output, but for the inherent benefits of the journey itself: the learning process, critical thinking, development of original arguments, and the unique personal touch on the work.

\clearpage
\nocite{*}
\printbibliography

\end{document}