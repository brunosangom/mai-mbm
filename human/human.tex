\documentclass[11pt,a4paper]{article}
\usepackage{enumitem}
\usepackage[utf8]{inputenc}
\usepackage{nameref}
\usepackage[margin=2.5cm]{geometry}
\usepackage[english]{babel}
\usepackage{csquotes}
\usepackage[style=numeric, sorting=none]{biblatex}
\usepackage{amsmath}
\usepackage{graphicx}
\usepackage[colorlinks=true, allcolors=blue]{hyperref}
\hypersetup{
    colorlinks,
    linkcolor=black,  
    urlcolor=blue,
    citecolor=blue
}
\setlength{\marginparwidth}{2cm}
\usepackage{comment}
\usepackage{float}
\usepackage{booktabs}

\bibliography{references}
\author{Bruno Sánchez Gómez}
\date{\today}


\begin{document}

\begin{titlepage}
    \centering
    \vspace*{2cm}
    {\Huge \bfseries MBM Essay 1: Human \par}
    \vspace{2cm}
    {\Large {\Huge Reverse-Engineering Brain Mechanisms through Interpretable Neural Networks} \par}
    \vspace{8cm}
    {\large \textbf{Bruno Sánchez Gómez} \par}
    \vfill
    {\large \today \par}
\end{titlepage}


\part{The Symbiotic Evolution of AI and Neuroscience: A Historical Perspective}

\section{Introduction}
\begin{itemize}
    \item Hook: The enduring quest to understand the human brain, the most complex computational machine known.
    \item Thesis Statement: The intertwined history of Artificial Intelligence and Neuroscience has led to a symbiotic relationship where advances in one field have consistently fueled progress in the other. This essay will trace this co-evolution, from early computational models inspired by neural processes to the contemporary use of interpretable neural networks to reverse-engineer the brain's intricate mechanisms, culminating in a review of the current state-of-the-art and future directions.
    \item Roadmap: Briefly outline the essay's structure, covering the historical review, the rise of deep learning, the advent of interpretable AI (XAI), and the application of these tools in modern neuroscience.
\end{itemize}

\section{The Dawn of AI and Early Neural Models (1940s-1960s)}
\begin{itemize}
    \item \textbf{The McCulloch-Pitts Neuron (1943):} The first mathematical model of a biological neuron. Discuss its significance as a foundational concept for both AI and computational neuroscience.
    \item \textbf{Hebb's Postulate and Learning (1949):} ``Neurons that fire together, wire together.'' Explain the Hebbian learning rule and its influence on early learning algorithms in neural networks.
    \item \textbf{The Perceptron (1958):} Frank Rosenblatt's invention and its initial promise for pattern recognition. Connect this to early models of sensory processing in the brain.
    \item \textbf{The ``AI Winter'' and its Thaw:} Briefly discuss the limitations identified by Minsky and Papert (1969) and the subsequent decline and eventual resurgence of neural network research.
\end{itemize}

\section{Connectionism and the Rise of Parallel Distributed Processing (1980s)}
\begin{itemize}
    \item \textbf{The PDP Group and the ``Connectionist'' Bible:} Discuss the impact of Rumelhart, Hinton, and McClelland's work.
    \item \textbf{Backpropagation and Multi-Layer Networks:} Explain the significance of the backpropagation algorithm in training more complex networks, allowing for the modeling of more sophisticated cognitive functions.
    \item \textbf{Early Applications in Cognitive Science:} Provide examples of how these models were used to simulate and understand phenomena like language acquisition, memory, and perception.
\end{itemize}

\section{The Deep Learning Revolution and its Impact on Neuroscience (2000s-Present)}
\begin{itemize}
    \item \textbf{The Unreasonable Effectiveness of Data and Computation:} The convergence of large datasets (e.g., ImageNet) and powerful GPUs.
    \item \textbf{Convolutional Neural Networks (CNNs) and the Visual Cortex:}
        \begin{itemize}
            \item Draw strong parallels between the hierarchical structure of CNNs and the organization of the primate visual stream (V1, V2, V4, IT).
            \item Discuss seminal work (e.g., Yamins \& DiCarlo) showing that CNNs trained on object recognition tasks develop representations remarkably similar to those found in the visual cortex.
        \end{itemize}
    \item \textbf{Recurrent Neural Networks (RNNs) and Sequential Processing:}
        \begin{itemize}
            \item Explain the architecture of RNNs (including LSTMs and GRUs) and their suitability for modeling time-series data.
            \item Connect this to the brain's processing of language, motor sequences, and decision-making over time.
        \end{itemize}
\end{itemize}

\clearpage

\part{The Black Box Dilemma and the Rise of Interpretable AI (XAI)}

\section{The ``Black Box'' Problem in Deep Learning}
\begin{itemize}
    \item \textbf{Defining the Challenge:} While deep neural networks achieve impressive performance, their internal workings are often opaque and difficult to understand.
    \item \textbf{Implications for Scientific Discovery:} In the context of neuroscience, a ``black box'' model that mimics brain function without revealing *how* it does so offers limited scientific insight. The goal is not just to replicate but to understand the underlying principles.
\end{itemize}

\section{A Taxonomy of Interpretable AI Methods}
\begin{itemize}
    \item \textbf{Post-hoc vs. Intrinsically Interpretable Models:} Differentiate between methods that analyze a trained model and those that are designed to be transparent from the outset.
    \item \textbf{Feature Attribution Methods:}
        \begin{itemize}
            \item \textbf{Saliency Maps:} Visualizing which input features (e.g., pixels in an image) are most important for a model's prediction.
            \item \textbf{LIME (Local Interpretable Model-agnostic Explanations) and SHAP (SHapley Additive exPlanations):} Explain how these methods build simpler, interpretable local models to approximate the behavior of the complex ``black box'' model.
        \end{itemize}
    \item \textbf{Model-based Interpretation:}
        \begin{itemize}
            \item \textbf{Analyzing Network Activations and Representations:} Techniques for visualizing and understanding the features learned by different layers of a network.
            \item \textbf{Causal Intervention and ``Ablation'' Studies:} Simulating lesion studies in neuroscience by deactivating specific neurons or connections in the network to observe the effect on its output.
        \end{itemize}
\end{itemize}

\section{Interpretable AI in Action: Bridging the Gap to Neuroscience}
\begin{itemize}
    \item \textbf{Reverse-Engineering Sensory Systems:}
        \begin{itemize}
            \item \textbf{Vision:} Beyond object recognition, how XAI helps us understand how CNNs represent texture, shape, and motion, and how this compares to neural data from different visual areas.
            \item \textbf{Audition:} Using deep learning models to understand the neural coding of sound, from the cochlea to the auditory cortex.
        \end{itemize}
    \item \textbf{Decoding and Encoding Models:} Using interpretable models to predict neural activity from stimuli (encoding) and to decode mental states or perceived stimuli from neural recordings (decoding).
    \item \textbf{Understanding Higher Cognitive Functions:}
        \begin{itemize}
            \item \textbf{Language:} Analyzing the representations within large language models (LLMs) like BERT and GPT, and comparing them to the brain's language processing centers (e.g., Broca's and Wernicke's areas).
            \item \textbf{Decision-Making and Reinforcement Learning:} How interpretable reinforcement learning models can shed light on the neural circuits involved in reward, planning, and action selection.
        \end{itemize}
\end{itemize}

\clearpage

\part{The State of the Art and the Future of Brain-Inspired AI}

\section{Current Frontiers in Interpretable Neural Networks for Neuroscience}
\begin{itemize}
    \item \textbf{Generative Models and ``In Silico'' Experiments:} Using generative adversarial networks (GANs) and other generative models to create stimuli that maximally activate specific neurons or brain regions, allowing for more targeted experiments.
    \item \textbf{Beyond Supervised Learning:} The role of self-supervised and unsupervised learning in creating models that learn more brain-like representations without requiring massive labeled datasets.
    \item \textbf{The Rise of ``Neuro-AI'':} The growing field of research that explicitly aims to build AI systems based on principles from neuroscience, creating a virtuous cycle of discovery.
    \item \textbf{Thinking LLMs and Simulating Thought Processes:} Discussing the emerging use of large language models to model and understand human-like reasoning, planning, and problem-solving.
\end{itemize}

\section{Challenges, Limitations, and Ethical Considerations}
\begin{itemize}
    \item \textbf{The ``Simile'' vs. ``Model'' Distinction:} Emphasize that even the most brain-like ANNs are still simplifications. Discuss the key biological details they often omit (e.g., dendritic computation, neuromodulation).
    \item \textbf{The Dangers of Over-interpretation:} The risk of drawing premature or overly simplistic conclusions about the brain based on analogies with AI models.
    \item \textbf{Data Privacy and Neuromarketing:} Briefly touch on the ethical implications of being able to decode brain states with increasing accuracy.
\end{itemize}

\section{Conclusion: The Future of a Fruitful Partnership}
\begin{itemize}
    \item \textbf{Recap of the Main Arguments:} Summarize the historical co-evolution and the current state of synergy between AI and neuroscience.
    \item \textbf{Future Outlook:} Project how this interdisciplinary collaboration will continue to unravel the complexities of the brain and, in turn, inspire more general and capable artificial intelligence. The ultimate goal: a unified theory of intelligence, both biological and artificial.
    \item \textbf{Final Thought-Provoking Statement:} Reiterate the profound potential of this research to not only advance science but also to fundamentally alter our understanding of ourselves.
\end{itemize}

\clearpage
\nocite{*}
\printbibliography%[title=References]

\end{document}